% ----------------------------------------------------------------------------------------
% SECTION TITLE
% ----------------------------------------------------------------------------------------

\cvsection{Projects}

% ----------------------------------------------------------------------------------------
% SECTION CONTENT
% ----------------------------------------------------------------------------------------

\begin{cventries}

  % ------------------------------------------------
  \cventry
  {Undergraduate Project, Prof. Amey Karkare}
  {{\entrytitlestyle{Tipsy: Tool to provide tips and corrections for MOOC submissions}}
    {}}
  {IIT Kanpur}
  {Jan. 2017 - Apr. 2017}
  {
    \begin{cvitems}
    \item Created a tool in Scala to parse, analyze and classify C programs from large programming courses, to help provide suggestions and tips to weak students.
    \item Reduced C programs to a linear high level representation, which was later used for finding shortest distance between 2 programs.
    \item Classified programs to provide suggestions to students based on programs similiar to their submission.
    \end{cvitems}
    \vspace{-5mm}
  }

  \cventry
  {Course Project, Prof. Amey Karkare}
  {\href{https://github.com/pallavagarwal07/amigo.git}{\entrytitlestyle{Amigo: A 4-stage x64 Compiler for Golang}}
    \ \ \ \normalfont\href{https://github.com/pallavagarwal07/amigo}
    {}}
  {IIT Kanpur}
  {Jan. 2017 - Apr. 2017}
  {
    \begin{cvitems}
    \item Implemented a compiler for a fully functional subset of the Go language, in C++ and Python.
    \item Used flex and bison to obtain an AST, which is later translated to a x64 assembly.
    \item Implemented pointers, multiple return values, deeply nested arrays, among other features; along with some low level optimizations.
    \end{cvitems}
  }

  \cventry
  {Course Project, Prof. Piyush Kurur and
    Prof. Satyadev Nandakumar}
  {\href{https://github.com/pclubiitk/puppy-love}{\entrytitlestyle{Anonymous
        and private pair matching platform}
      \ \ \ \normalfont{\href{https://github.com/pclubiitk/puppy-love}
      {acehack.org/puppy}}}}
  {IIT Kanpur}
  {Nov. 2016 - Feb. 2017}
  {
    \begin{cvitems}
      \item Designed and implemented an algorithmic platform
        for anonymous pair/couple matching.
      \item Ensures that users' choices are not made known even to the
        server admin.
      \item Used Diffie-Hellman like token exchange over an
        honest-but-curious server backend, asymmetric encryption to
        ensure confidentiality and fairness even during matching.
    \end{cvitems}
  }

  \cventry
  {Undergraduate Project, Prof. Sandeep Shukla}
  {\href{https://github.com/netsecIITK/moVi}{\entrytitlestyle{moVi: Mobile Video Chat Protocol}}
    \ \ \ \normalfont\href{https://github.com/netsecIITK/moVi}
    {github.com/netsecIITK/movi}}
  {IIT Kanpur}
  {Sept. 2016 - Nov. 2016}
  {
    \begin{cvitems}
    \item Developed a client for video communication
      akin to \href{https://mosh.org/}{Mosh (mobile shell)}.
    \item Used UDP to set up a connection-less and secure channel,
      persistent across network IP and location changes.
    \item Implemented State Synchronization Protocol, UDP Hole
      Punching, and dynamic tweaking of video quality.
    \end{cvitems}
  }

  \cventry
  {Member, Team Robocon IIT Kanpur, Prof. Bhaskardas Gupta}
  {ABU Robocon 2015, Badminton playing robots}
  {IIT Kanpur}
  {Oct. 2014 - Mar. 2015}
  {
    \begin{cvitems}
    \item Programmed and built 2 semi-autonomous robots
      capable of playing badminton on a standard size court.
    \item Used image processing with OpenCV to detect the shuttle
      and predict the trajectory.
    \item Used Kinect and Stereo Vision to get depth of
      field. Programmed the robot using Arduino run by Odroid.
    \item Finished 11th among 85 teams all over India.
    \end{cvitems}
  }

  % ------------------------------------------------
\end{cventries}

%%% Local Variables:
%%% mode: latex
%%% TeX-master: "../resume_twopage"
%%% End:
