% ----------------------------------------------------------------------------------------
% SECTION TITLE
% ----------------------------------------------------------------------------------------

\cvsection{Projects}

% ----------------------------------------------------------------------------------------
% SECTION CONTENT
% ----------------------------------------------------------------------------------------

\begin{cventries}

  % ------------------------------------------------
  \cventry
  {Undergraduate Project, Prof. Amey Karkare}
  {{\entrytitlestyle{Tipsy: Tool to provide tips and corrections for C programs en masse}}
    {}}
  {IIT Kanpur}
  {Jan. 2017 - Apr. 2017}
  {
    \begin{cvitems}
    \item Developed a tool in Scala to parse, analyze and classify C
      programs from large programming courses, to help provide
      suggestions and tips to weak students.
    \item Reduced C programs to a linear high level representation. Used it to find similar features between programs, thus allowing discovering corrections in a program using reference programs.
    \item Classified programs to provide suggestions to students based on programs similiar to their submission.
      \item \textbf{In proceedings} of the 19th International Conference on
        Artificial Intelligence in Education, London.
    \end{cvitems}
    \vspace{-5mm}
  }

  \cventry
  {Course Project, Prof. Piyush Kurur and
    Prof. Satyadev Nandakumar}
  {\href{https://github.com/pclubiitk/puppy-love}{\entrytitlestyle{Anonymous and private couple matching platform}}}
  {IIT Kanpur}
  {Nov. 2016 - Apr. 2017}
  {
    \begin{cvitems}
      \item Designed an algorithm and implemented a platform
        for anonymous pair/couple matching.
      \item Ensures that users' choices are not made known, even to the
        server admin.
      \item Used Diffie-Hellman like token exchange over an
        honest-but-curious server backend, along with asymmetric encryption, to
        ensure confidentiality and fairness even while matching people.
      \item Deployed on campus for a week, with 1800+ users and 45 matches.
    \end{cvitems}
  }


  \cventry
  {Course Project, Prof. Amey Karkare}
  {\href{https://github.com/pallavagarwal07/amigo.git}{\entrytitlestyle{Amigo: A 4-stage x64 Compiler for Golang}}
    \ \ \ \normalfont\href{https://github.com/pallavagarwal07/amigo}
    {}}
  {IIT Kanpur}
  {Jan. 2017 - Apr. 2017}
  {
    \begin{cvitems}
    \item Implemented a compiler for a fully functional subset of the Go language, in C++ and Python.
    \item Used flex and bison to obtain an AST, which is later translated to a x64 assembly.
    \item Implemented pointers, multiple return values, deeply nested arrays, with some assembly level optimizations.
    \end{cvitems}
  }

  \cventry
  {Undergraduate Project, Prof. Sandeep Shukla}
  {\href{https://github.com/netsecIITK/moVi}{\entrytitlestyle{moVi: Video Application for mobile networks}}
    \ \ \ \normalfont\href{https://github.com/netsecIITK/moVi}
    {}}
  {IIT Kanpur}
  {Sept. 2016 - Nov. 2016}
  {
    \begin{cvitems}
    \item Developed a Linux client for video chatting,
      based on \href{https://mosh.org/}{Mosh (mobile shell)}.
    \item Used UDP to set up a connection-less and secure channel,
      persistent across network IP and location changes.
    \item Implemented State Synchronization Protocol, UDP Hole
      Punching, and dynamic tweaking of video quality.
    \end{cvitems}
  }

  \cventry
  {Member, Team Robocon IIT Kanpur, Prof. Bhaskardas Gupta}
  {ABU Robocon 2015, Badminton playing robots}
  {IIT Kanpur}
  {Oct. 2014 - Mar. 2015}
  {
    \begin{cvitems}
    \item Programmed and built 2 semi-autonomous robots
      capable of playing badminton on a standard size court.
    \item Used OpenCV, Kinect and Stereo Vision to detect the shuttle,
      and predict its trajectory.
    \item Finished 11th among 85 teams all over India.
    \end{cvitems}
  }

  % ------------------------------------------------
\end{cventries}

%%% Local Variables:
%%% mode: xelatex
%%% TeX-master: "../resume_twopage.tex"
%%% End:
